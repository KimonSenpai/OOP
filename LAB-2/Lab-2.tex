\documentclass[a4paper,12pt]{article}
\usepackage{amsmath,amssymb,amsfonts,amsthm}
\usepackage{tikz}
\usepackage [utf8] {inputenc}
\usepackage [T2A] {fontenc} 
\usepackage[russian]{babel}
% Так ссылки в PDF будут активны
\usepackage[unicode]{hyperref}
\usepackage{ textcomp }
\usepackage{indentfirst}


%===============================================================================

% Title settings for Russian language in amsart

%===============================================================================

\makeatletter
\def\@settitle{\begin{center}%
		\baselineskip14\p@\relax
		\bfseries\scshape
		\@title
	\end{center}%
}
\makeatother

%===============================================================================

% Theorem styles

%===============================================================================
\usepackage{amsxtra}
\usepackage{enumitem, fancyref, theoremref}
\usepackage{amsrefs, latexsym, pstricks, mathtext}
\usepackage{hyperref, caption, listings}

\hypersetup{
	colorlinks=true,
	linkcolor=blue,
	filecolor=magenta,      
	urlcolor=cyan,
	pdftitle={Overleaf Example},
	pdfpagemode=FullScreen,
}

\urlstyle{same}

\renewenvironment{proof}{{\bfseries\itshape Доказательство.}}{\begin{flushright}
		$\Box$
\end{flushright}}

\newenvironment{sol}{{\bfseries\itshape Решение.}}{\begin{flushright}
		$\Box$
\end{flushright}}

\theoremstyle{plain}
\newtheorem*{quest}{Вопрос}
\newtheorem{prop}{Предложение}
\newtheorem{theorem}{Теорема}[section]
\newtheorem{lemma}{Лемма}[section]

\theoremstyle{remark}
\newtheorem{conseq}{Вывод}
\newtheorem{rem}{Замечание}

\theoremstyle{definition}
\newtheorem{definition}{Определение}

\newtheoremstyle{problem}
{\topsep}   % ABOVESPACE
{\topsep}   % BELOWSPACE
{\normalfont}  % BODYFONT
{0pt}       % INDENT
{\bfseries} % HEADFONT
{.}         % HEADPUNCT
{5pt plus 1pt minus 1pt} % HEADSPACE
{}          % CUSTOM-HEAD-SPEC
\newtheorem{problem}{Задача}

%===============================================================================



% вы сможете вставлять картинки командой \includegraphics[width=0.7\textwidth]{ИМЯ ФАЙЛА}
% получается подключать, как минимум, файлы .pdf, .jpg, .png.
\usepackage{graphicx}
% Если вы хотите явно указать поля:
\usepackage[margin=1in]{geometry}
% Или если вы хотите задать поля менее явно (чем больше DIV, тем больше места под текст):
% \usepackage[DIV=10]{typearea}

\usepackage{fancyhdr}

\newcommand{\bbR}{\mathbb R}%теперь вместо длинной команды \mathbb R (множество вещественных чисел) можно писать короткую запись \bbR. Вместо \bbR вы можете вписать любую строчку букв, которая начинается с '\'.
\newcommand{\eps}{\varepsilon}
\newcommand{\bbN}{\mathbb N}
\newcommand{\dif}{\mathrm{d}}

\newtheorem{Def}{Definition}

%===============================================================================
%%%%%%%%%%%%%%%%%%%%%%%%%%%% Block schema %%%%%%%%%%%%%%%%%%%%%%%%%%%%%%%%%%%%%%
%===============================================================================

\usepackage{tikz}

\usetikzlibrary{shapes, arrows}

\tikzstyle{decision} = [
	diamond,
	draw,
	fill = green!20,
	text width = 6em,
	text badly centered,
	node distance = 2cm,
	inner sep = 0pt
]
\tikzstyle{block} = [
	rectangle,
	draw,
	fill = blue!20,
	text width = 8em,
	text centered,
	rounded corners,
	minimum height = 2em
]
\tikzstyle{line} = [
	draw,
	-latex'
]
\tikzstyle{cloud} = [
	draw,
	ellipse,
	fill = red!20,
	node distance = 3cm,
	minimum height = 2em
]

\tikzstyle{blanc} = [
		draw=white,
		rectangle,
		fill = white,
		minimum width = 0em,
		minimum height = 0em
]

%================================================================================


\pagestyle{fancy}
\makeatletter % сделать "@" "буквой", а не "спецсимволом" - можно использовать "служебные" команды, содержащие @ в названии
\fancyhead[L]{\footnotesize \@title}%Это будет написано вверху страницы слева
\fancyhead[R]{\footnotesize <<Физтех-лицей>> им. П. Л. Капицы}
\fancyfoot[L]{\footnotesize \@author}%имя автора будет написано внизу страницы слева
\fancyfoot[R]{\@date}%номер страницы —- внизу справа
\fancyfoot[C]{\thepage}%по центру внизу страницы пусто

\renewcommand{\maketitle}{%Настройка заголовка
	\noindent{\bfseries\scshape\large\@title\ \mdseries\upshape}\par
	\noindent {\large\itshape Author: \@author}
	\vskip 2ex}
\makeatother
\def\dd#1#2{\frac{\partial#1}{\partial#2}}
\def\ssum#1#2{\sum\limits_{#1}^{#2}}
\def\l{\langle}
\def\r{\rangle}

\graphicspath{{images//}}

\hypersetup{
	pdfauthor={Kim Zyong}
}

\usepackage{multirow}
\usepackage{colortbl}





\title{Отчёт по заданию к семинару 1\\Вариант 22}
\author{Ким Зыонг ИДз-22-20} 
\date{\today}

\begin{document}
	\maketitle
	
	\section{Условие задачи}
	
	Для каждой строки матрицы $A (4\times5)$ вычислить сумму и количество
	отрицательных элементов, а для каждой строки матрицы $B (3\times 7)$ — сумму и
	количество элементов, значения которых меньше 5.
	
	\section{Материалы}
	
	Все материалы проекта доступны по ссылке: \url{https://github.com/KimonSenpai/OOP/tree/main/LAB-1}
	
	Основные файлы:
	\begin{enumerate}
		\item Lab-1.cpp --- файл с основной программой;
		
		\item Prog.exe --- скомпилированный файл Lab-1.cpp;
		
		\item gen-test-A.py, gen-test-B.py --- программы, генерирующие тестовые матрицы A и B (по 20 штук). Они помещаются в соответствующие папки в виде 2 файлов. Файл ``<номер>'' содержит саму матрицу, а файл ``<номер>.a'' -- ожидаемый вывод программы;
		
		\item check-A.py, check-B.py --- проверяют правильность работы программы Prog.exe на тестовых данных соответствующего типа и записывают результат в следующие файлы;
		
		\item ResoultA.txt, ResaultB.txt --- содержат вердикты проверки по каждому из тестов;
		
		\item EXEC.cmd --- при запуске осуществляет компиляцию программы, генерацию тестов и проверку. Для работы требует наличия утилит g++ и python. Также они должны быть прописаны в переменной PATH;
		
		\item Lab-1.tex --- исходник данного документа.
	\end{enumerate}
	\section{Блок-схема основного алгоритма}
	
	В данной схеме некоторые объекты переименованы для экономии места и наглядности:
	
	\begin{enumerate}
		\item $matr\rightarrow m$;
		
		\item $sumLess\rightarrow sL$;
		
		\item $countLess\rightarrow cL$.
	\end{enumerate}
	
	\begin{tikzpicture}[node distance = 1.5cm, auto]
		\node [cloud] (start) {Начало};
		\node [block, below of = start] (ph1) {$i = 0$};
		\node [decision, below of = ph1] (c1) {$i < M$};
		\node [block, below of = c1, node distance = 2.5cm] (ph2) {$sL[i] = 0$};
		\node [block, below of = ph2] (ph3) {$cL[i] = 0$};
		\node [block, below of = ph3] (ph4) {$j = 0$};
		\node [decision, below of = ph4] (c2) {$j < N$};
		\node [decision, below of = c2, node distance = 3.5cm] (c3) {$m[i][j] < q$};
		\node [block, below of = c3, node distance = 2.5cm] (ph5) {$sL[i] += m[i][j]$};
		\node [block, below of = ph5] (ph6) {$cL[i] += 1$};
		\node [block, below of = ph6] (ph7) {$j += 1$};
		\node [block, below of = ph7] (ph8) {$i += 1$};
		\node [cloud, below of = ph8, node distance = 1.5cm] (finish) {Конец};
		
		\node [blanc, right of = ph5, node distance = 3cm] (b1) {};
		\node [blanc, left of = ph5, node distance = 3cm] (b2) {};
		\node [blanc, left of = b2, node distance = 2cm] (b3) {};
		\node [blanc, right of = b1, node distance = 2cm] (b4) {};
		\node [blanc, left of = b3, node distance = 2cm] (b5) {};
		
		\path [line] (start) -- (ph1);
		\path [line] (ph1) -- (c1);
		\path [line] (c1) -- node[near start]{Да} (ph2);
		\path [line] (ph2) -- (ph3);
		\path [line] (ph3) -- (ph4);
		\path [line] (ph4) -- (c2);
		\path [line] (c2) -- node[near start]{Да} (c3);
		\path [line] (c3) -- node[near start]{Да} (ph5);
		\path [line] (ph5) -- (ph6);
		\path [line] (ph6) -- (ph7);
		
		\path [line] (ph7) -| (b1);
		\path [line] (b1) |- (c2);
		
		\path [line] (c3) -| node[near start]{Нет} (b2);
		\path [line] (b2) |- (ph7);
		
		\path [line] (c2) -| node[near start]{Нет} (b3);
		\path [line] (b3) |- (ph8);
		
		\path [line] (ph8) -| (b4);
		\path [line] (b4) |- (c1);
		
		\path [line] (c1) -| node[near start]{Нет} (b5);
		\path [line] (b5) |- (finish);
	\end{tikzpicture}
	
	
	
\end{document}